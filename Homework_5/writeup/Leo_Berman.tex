\documentclass{article}
\pagenumbering{arabic}
\newcommand{\newCommandName}{text to insert}
% Pages and colors used for the cover page
\usepackage{tikz-page}
\usepackage{url}
\usepackage{lastpage}

% Used for the math and code
\usepackage{amsmath}
\usepackage{listings}
\usepackage{pythonhighlight} % Got this from github! https://github.com/olivierverdier/python-latex-highlighting/blob/master/pythonhighlight.sty

% Used for the pictures
\usepackage{float}
\usepackage{graphicx}

% For the table
\usepackage[utf8]{inputenc}
\usepackage{multirow}
\usepackage{colortbl}
\usepackage{array}
\usepackage{tabularray}

\usepackage[
  top=2cm,
  bottom=2cm,
  left=3cm,
  right=2cm,
  headheight=17pt, % as per the warning by fancyhdr
  includehead,includefoot,
  heightrounded, % to avoid spurious underfull messages
]{geometry} 

% Page numbers/header
\usepackage{fancyhdr}
\definecolor{brickred}{rgb}{0.8, 0.25, 0.33}
\definecolor{cobalt}{rgb}{0.0, 0.28, 0.67}
\definecolor{cadetgrey}{rgb}{0.57, 0.64, 0.69}

% Defining the text box being used for DEPT OF ENG
\tikzset{
        secnode/.style={
                minimum height = .16in,
                minimum width = 4.16in,
                inner xsep = 2pt,
                anchor=north east,
                draw=cadetgrey,
                fill=white,
                text=brickred,
                },
        }


         
\pagestyle{plain}
\renewcommand{\headrulewidth}{0pt}
\begin{document}


% Put name data and assignment number here
\newcommand\personaldate{March 23, 2024}
\newcommand\myname{Leo Berman/Shane McNicholas}
\newcommand\myemail{leo.berman@temple.edu/shane.mcnicholas@temple.edu}
\newcommand\hwnum{05}
\newcommand\mynameabbrev{L. Berman\S. McNicholas}
\newcommand\assignmenttitle{Dynamic Programming}
\newcommand\yourclass{ECE 8527: Machine Learning and Pattern Recognition}
\begin{titlepage}
	% Drawing the border and the text box 
	\newcommand{\tikzpagelayout}{
		\draw[line width = .04in,
			color = cobalt]
		($(current page.north west)+(1in,-1in)$)
		rectangle ($(current page.south east)+(-.625in,1in)$);

		\draw[line width = .04in,
			color = brickred]
		($(current page.north west)+(.92in,-.92in)$)
		rectangle ($(current page.south east)+(-.705in,1.08in)$);
		\node[secnode] at ($(current page.north west)+(6in,-.875in)$) {\small{\textbf{DEPARTMENT OF ELECTRICAL AND COMPUTER ENGINEERING}}};
	}

	\begin{center}
		\large{Homework Assignment No. \hwnum:}\break
		\break
		\large{\textbf{HW No. \hwnum: \assignmenttitle}}\break
		\break
		\large{submitted to \:}\break
		\break
		\large{Professor Joseph Picone}\break
		\large{ECE 8527: Introduction to Pattern Recognition and Machine Learning}\break
		\large{Temple University}\break
		\large{College of Engineering}\break
		\large{1947 North 12th Street}\break
		\large{Philadelphia, Pennsylvania 19122}\break
		\break
		\large{\personaldate}\break
		\break
		\large{prepared by: }\break
		\break
		\large{\myname}\break
		\large{Email: \myemail}
	\end{center}
\end{titlepage}

\newpage
\pagestyle{fancy}
\fancyhead{}
\fancyfoot{}
\fancyhead[R,EH]{Page \thepage\ of \pageref{LastPage}}
\fancyhead[L,EH]{\mynameabbrev: HW \# \hwnum}
\fancyfoot[L,EF]{\yourclass}
\fancyfoot[R,EF]{\personaldate}
\renewcommand{\thesection}{\Alph{section}.}

\section{\MakeUppercase{Testing Approach}}
\flushleft{In order to test our approach, we decided to test it against all of the test cases provided to us. We copied all of the tests over and used a combination of python and command line tools in order to parse each test into a file that the program understands. We then iterated through each test for the program and compared the outputs of our program versus the ouput of the current system.}
\section{\MakeUppercase{Example Results}}
\begin{flushleft}
Our code got every test correct. However, it got different solutions on a few tests:
\break\break

A transaction requires approval of a majority of A SHARE A LOT of the LATIN AMERICA\break
THE transaction requires approval of a majority of THE SHARES of the HOLDERS NOT AFFILIATED WITH MR. ICAHN\break
\textbf{Control Results:}\break
\textbf{Substitutions}: 5 \textbf{Insertions}: 4 \textbf{Deletions}: 2 \textbf{Total}: 11\break
\textbf{Our Results:}\break
\textbf{Substitutions}: 9 \textbf{Insertions}: 2 \textbf{Deletions}: 0 \textbf{Total}: 11\break

the company RAN INTO ITS shares tendered AND the lowest PRICED BELOW THREE to ELIMINATE that amount OF all shares THE AREA\break
the company THEN ACCEPTS THE shares tendered AT the lowest PRICE NEEDED to REACH ITS GOAL THEN PAYS that amount FOR all shares IT PURCHASES\break
\textbf{Control Results:}\break
\textbf{Substitutions}: 10 \textbf{Insertions}: 4 \textbf{Deletions}: 1 \textbf{Total}: 15\break
\textbf{Our Results:}\break
\textbf{Substitutions}: 12 \textbf{Insertions}: 3 \textbf{Deletions}: 0 \textbf{Total}: 15\break

WHO ARE not to HEAD OF the END THE CANDIDATE WHO\break
WE'RE not PREPARED to BE ADVOCATES FOR the K. G. B.\break
\textbf{Control Results:}\break
\textbf{Substitutions}: 6 \textbf{Insertions}: 2 \textbf{Deletions}: 2 \textbf{Total}: 10\break
\textbf{Our Results:}\break
\textbf{Substitutions}: 10 \textbf{Insertions}: 0 \textbf{Deletions}: 0 \textbf{Total}: 10\break



\end{flushleft}

\section{\MakeUppercase{Summary}}
\flushleft{Overall, we can see that our program has a bias towards substitutions whereas the control algorithm has a more even distribution. There were 25 test cases that followed this pattern. This seems to be a difference in the alogrithms or perhaps in classifying some of the changes as 2 substitutions instead of an insertion and a deletion.}
\end{document}