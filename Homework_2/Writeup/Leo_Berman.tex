\documentclass{article}
\pagenumbering{arabic}
\newcommand{\newCommandName}{text to insert}
% Pages and colors used for the cover page
\usepackage{tikz-page}
\usepackage{url}
\usepackage{lastpage}

% Used for the math and code
\usepackage{amsmath}
\usepackage{listings}
\usepackage{pythonhighlight} % Got this from github! https://github.com/olivierverdier/python-latex-highlighting/blob/master/pythonhighlight.sty

% Used for the pictures
\usepackage{float}

\usepackage[
  top=2cm,
  bottom=2cm,
  left=3cm,
  right=2cm,
  headheight=17pt, % as per the warning by fancyhdr
  includehead,includefoot,
  heightrounded, % to avoid spurious underfull messages
]{geometry} 
% Page numbers/header
\usepackage{fancyhdr}
\definecolor{brickred}{rgb}{0.8, 0.25, 0.33}
\definecolor{cobalt}{rgb}{0.0, 0.28, 0.67}
\definecolor{cadetgrey}{rgb}{0.57, 0.64, 0.69}

% Defining the text box being used for DEPT OF ENG
\tikzset{
        secnode/.style={
                minimum height = .16in,
                minimum width = 4.16in,
                inner xsep = 2pt,
                anchor=north east,
                draw=cadetgrey,
                fill=white,
                text=brickred,
                },
        }


         
\pagestyle{plain}
\renewcommand{\headrulewidth}{0pt}
\begin{document}


% Put name data and assignment number here
\newcommand\personaldate{February 3, 2024}
\newcommand\myname{Leo Berman}
\newcommand\myemail{leo.berman@temple.edu}
\newcommand\hwnum{02}
\newcommand\mynameabbrev{L. Berman}
\newcommand\assignmenttitle{Bayesian Decision Theory}
\newcommand\yourclass{ECE 8527: Machine Learning and Pattern Recognition}
\begin{titlepage}
	% Drawing the border and the text box 
	\newcommand{\tikzpagelayout}{
		\draw[line width = .04in,
			color = cobalt]
		($(current page.north west)+(1in,-1in)$)
		rectangle ($(current page.south east)+(-.625in,1in)$);

		\draw[line width = .04in,
			color = brickred]
		($(current page.north west)+(.92in,-.92in)$)
		rectangle ($(current page.south east)+(-.705in,1.08in)$);
		\node[secnode] at ($(current page.north west)+(6in,-.875in)$) {\small{\textbf{DEPARTMENT OF ELECTRICAL AND COMPUTER ENGINEERING}}};
	}

	\begin{center}
		\large{Homework Assignment No. \hwnum:}\break
		\break
		\large{\textbf{HW No. \hwnum: \assignmenttitle}}\break
		\break
		\large{submitted to:}\break
		\break
		\large{Professor Joseph Picone}\break
		\large{ECE 8527: Introduction to Pattern Recognition and Machine Learning}\break
		\large{Temple University}\break
		\large{College of Engineering}\break
		\large{1947 North 12th Street}\break
		\large{Philadelphia, Pennsylvania 19122}\break
		\break
		\large{\personaldate}\break
		\break
		\large{prepared by: }\break
		\break
		\large{\myname}\break
		\large{Email: \myemail}
	\end{center}
\end{titlepage}

\newpage
\pagestyle{fancy}
\fancyhead{}
\fancyfoot{}
\fancyhead[R,EH]{Page \thepage\ of \pageref{LastPage}}
\fancyhead[L,EH]{\mynameabbrev: HW \# \hwnum}
\fancyfoot[L,EF]{\yourclass}
\fancyfoot[R,EF]{\personaldate}
\renewcommand{\thesection}{\Alph{section}.}
\section{DESCRIPTION OF THE TASK}
\begin{flushleft}
	\large{Briefly describe the general approach that you used to solve the problem(s). Show snippets of code and explain how this code works. Show results in tables comparing new results to your baselines.}\break
	\break
	\large{For examples of well-formatted documents please review these conference abstracts:}\break
	\break
	\url{https://isip.piconepress.com/publications/conference_presentations/2021/ieee_spmb/dpath/abstract_v22.docx}\break
	\url{https://isip.piconepress.com/publications/conference_presentations/2021/ieee_spmb/tueg/abstract_v10.docx}
\end{flushleft}
\section{Generate multivariate Gaussian distributions}
\large{Generate multivariate Gaussian distributions with a mean vector of zero and covariances matrices :}
\[
(A)\begin{bmatrix} 
        1 & 0\\
        0 & 1
\end{bmatrix}
(B)\begin{bmatrix} 
        5 & 0\\
        0 & 2
\end{bmatrix}
(C)\begin{bmatrix} 
        2 & 0\\
        0 & 5
\end{bmatrix}
(D)\begin{bmatrix} 
        1 & .5\\
        .5 & 1
\end{bmatrix}
(E)\begin{bmatrix} 
        1 & -.5\\
        -.5 & 1
\end{bmatrix}
(F)\begin{bmatrix} 
        5 & .5\\
        .5 & 2
\end{bmatrix}
(G)\begin{bmatrix} 
        5 & -5\\
        -.5 & 2
\end{bmatrix}
\]
\break\break
In order to generate these covariance's and plot their eigenvectors I used the following matlab script.
\begin{python}
        # Import relevant libraries
        import numpy
        import pandas
        import matplotlib.pyplot as plt

        # generate multivariate guassians
        def GMG(mean,cov,nelem):
            return numpy.random.multivariate_normal(mean,cov,nelem)
        
        # calculate eigenvectors
        def CE(inmat):
            return numpy.linalg.eig(inmat)
        
        def main():
            # number of elements
            number_elements = 5000
        
           # declare mean for all covariance matrixes
            mean = [0,0]
        
            # declare covariance matrixes
            cov_0 = [[1,0],[0,1]]
            cov_1 = [[5,0],[0,2]]
            cov_2 = [[2,0],[0,5]]
            cov_3 = [[1,.5],[.5,1]]
            cov_4 = [[1,-.5],[-.5,1]]
            cov_5 = [[5,.5],[.5,2]]
            cov_6 = [[5,-.5],[-.5,2]]
        
            # concatenate all covariance matrixes into a list
            cov_list = [cov_0,cov_1,cov_2,cov_3,cov_4,cov_5,cov_6]
        
            # declare list for points to go into
            points_list = []
        
            # declare list of eigenvectors and eIgenvalues
            eigenvectors_list = []
            eigenvalues_list = []
        
            # Colors list
            colors_list = ['r','b','y','g','o']
        
            # Generate guassians for allcovariance matrixes
            for i in range(len(cov_list)):
                # Generate Guassian points
                points_list.append(GMG(mean,cov_list[i],number_elements))
                
                # Plot the points
                plt.plot(points_list[i][:,0], points_list[i][:,1], '.', alpha = 0.5, zorder = 0)
                
                # Calculate covariance matrix eigenvectors and eigenvalues
                eigenvalues,eigenvectors = CE(cov_list[i])
                eigenvectors_list.append(eigenvectors)
                eigenvalues_list.append(eigenvalues)
        
                # Plot the eigenvectors
                for j in range(len(eigenvectors)):
                    print(eigenvalues[j])     
                    plt.quiver(*mean, *(eigenvectors[:,j]*eigenvalues[j]), scale = 18, color = colors_list[j], zorder = 10)
                
                # Plot the rest of the points and save it as a png
                plt.xlim(-10,10)
                plt.ylim(-10,10)
                plt.grid()
                #plt.show()
                plotname = "Cov_" + str(i) + ".png"
                plt.savefig(plotname)
                plt.clf()
                plt.cla()
               
        main()

\end{python}
\section{SUMMARY}
\large{Briefly describe what you learned from this assignment and ways you could improve your solutions.}
\end{document}