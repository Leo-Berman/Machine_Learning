\documentclass{article}
\pagenumbering{arabic}
\newcommand{\newCommandName}{text to insert}
% Pages and colors used for the cover page
\usepackage{tikz-page}
\usepackage{url}
\usepackage[
  top=2cm,
  bottom=2cm,
  left=3cm,
  right=2cm,
  headheight=17pt, % as per the warning by fancyhdr
  includehead,includefoot,
  heightrounded, % to avoid spurious underfull messages
]{geometry} 
% Page numbers/header
\usepackage{fancyhdr}
\definecolor{brickred}{rgb}{0.8, 0.25, 0.33}
\definecolor{cobalt}{rgb}{0.0, 0.28, 0.67}
\definecolor{cadetgrey}{rgb}{0.57, 0.64, 0.69}

% Defining the text box being used for DEPT OF ENG
\tikzset{
        secnode/.style={
                minimum height = .16in,
                minimum width = 4.16in,
                inner xsep = 2pt,
                anchor=north east,
                draw=cadetgrey,
                fill=white,
                text=brickred,
                },
        }


         
\pagestyle{plain}
\renewcommand{\headrulewidth}{0pt}
\begin{document}
        
        
        % Put name data and assignment number here
        \newcommand\personaldate{February 3, 2024}
        \newcommand\myname{John Smith}
        \newcommand\myemail{jsmith@temple.edu}
        \newcommand\hwnum{XX}
        \newcommand\mynameabbrev{J. Smith}
        \newcommand\assignmenttitle{Title of the Assignment}

        \begin{titlepage}
                % Drawing the border and the text box 
                \newcommand{\tikzpagelayout}{
                \draw[line width = .04in,
                color = cobalt]
                ($(current page.north west)+(1in,-1in)$)
                rectangle ($(current page.south east)+(-.625in,1in)$);
        
                \draw[line width = .04in,
                color = brickred]
                ($(current page.north west)+(.92in,-.92in)$)
                rectangle ($(current page.south east)+(-.705in,1.08in)$);
                \node[secnode] at ($(current page.north west)+(6in,-.875in)$) {\small{\textbf{DEPARTMENT OF ELECTRICAL AND COMPUTER ENGINEERING}}};
                }

                \begin{center}
                        \large{Homework Assignment No. \hwnum:}\break
                        \break
                        \large{\textbf{HW No. \hwnum: \assignmenttitle}}\break
                        \break
                        \large{submitted to:}\break
                        \break
                        \large{Professor Joseph Picone}\break
                        \large{ECE 8527: Introduction to Pattern Recognition and Machine Learning}\break
                        \large{Temple University}\break
                        \large{College of Engineering}\break
                        \large{1947 North 12th Street}\break
                        \large{Philadelphia, Pennsylvania 19122}\break
                        \break
                        \large{\personaldate}\break
                        \break
                        \large{prepared by: }\break
                        \break
                        \large{\myname}\break
                        \large{Email: \myemail}
                \end{center}
        \end{titlepage}

        \newpage
        \pagestyle{fancy}
        \fancyhead{}
        \
        \fancyhead[R,EH]{Page \thepage of 1}%\pageref{LastPage}}
        \fancyhead[L,EH]{\mynameabbrev: HW \# \hwnum}
        \renewcommand{\thesection}{\Alph{section}.}
        \section{DESCRIPTION OF THE TASK}
        \begin{flushleft}
        Briefly describe the general approach that you used to solve the problem(s). Show snippets of code and explain how this code works. Show results in tables comparing new results to your baselines.\break
        \break
        For examples of well-formatted documents please review these conference abstracts:\break
        \break
        \url{https://isip.piconepress.com/publications/conference_presentations/2021/ieee_spmb/dpath/abstract_v22.docx}\break
        \url{https://isip.piconepress.com/publications/conference_presentations/2021/ieee_spmb/tueg/abstract_v10.docx}
        \end{flushleft}
        \section{DESCRIPTION OF THE TASK}
        Repeat the same process for the next task.
        \section{SUMMARY}
        Briefly describe what you learned from this assignment and ways you could improve your solutions.
\end{document}